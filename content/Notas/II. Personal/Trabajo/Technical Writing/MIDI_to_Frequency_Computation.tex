\documentclass[11pt]{article}
\usepackage[utf8]{inputenc}
\usepackage{geometry}
\usepackage{amsmath}
\usepackage{listings}
\usepackage{xcolor}

\geometry{margin=1in}

% Code snippet styling
\definecolor{codegray}{rgb}{0.5,0.5,0.5}
\definecolor{codeblue}{rgb}{0.1,0.1,0.9}
\lstset{
    language=C,
    basicstyle=\ttfamily\small,
    keywordstyle=\color{codeblue},
    commentstyle=\color{codegray},
    breaklines=true,
    frame=single
}

\title{\textbf{Technical Reference: MIDI to Frequency Conversion}}
\author{Technical Writing Portfolio | Audio Engineering}
\date{2025}

\begin{document}

\maketitle

\section{Introduction}
In digital audio workstation (DAW) development, converting MIDI note numbers into Hertz (Hz) is a fundamental computational task. This document outlines the mathematical relationship between semitones and frequency, based on the Equal Temperament scale where the A4 reference pitch is set to 440 Hz.

\section{Mathematical Principles}
The relationship between frequencies in equal temperament is logarithmic. Every octave represents a doubling of frequency, divided into 12 semitones.

\subsection{The Semitone Ratio}
The constant used to calculate the interval between adjacent semitones is the twelfth root of two ($2^{1/12}$):
\begin{equation}
a = \sqrt[12]{2} \approx 1.059463094359
\end{equation}

\subsection{The Conversion Formula}
To calculate the frequency $f$ for any MIDI note $n$, where MIDI note 69 corresponds to A4 (440 Hz), the formula is:
\begin{equation}
f(n) = 440 \cdot 2^{\frac{n-69}{12}}
\end{equation}

\section{Software Implementation (C)}
The following C snippet demonstrates the practical implementation of this formula for use in a real-time audio oscillator.

\begin{lstlisting}[caption={C implementation of MIDI to Frequency conversion}]
#include <stdio.h>
#include <math.h>

double midiToFrequency(int midiNote) {
    // Reference pitch for A4 is 440Hz, which is MIDI note 69
    double referencePitch = 440.0;
    int referenceNote = 69;

    // Calculate frequency using the pow function
    return referencePitch * pow(2.0, (midiNote - referenceNote) / 12.0);
}
\end{lstlisting}

\section{Data Representation: 2's Complement}
In 16-bit digital audio systems, the amplitude of the signal is stored using the \textbf{2's Complement} binary system. This allows for the representation of both positive and negative voltage values, essential for reproducing the sinusoidal oscillations of sound waves within a fixed bit-depth ($-32,768$ to $+32,767$).

\end{document}